\chapter{Project Work}\label{final}

\textbf{\large Deep Learning Models for Handwritten
English Character Recognition:}

\\This section describes the startegy used for implementing character recognition from a given database. Before further proceeding, you are required
to know basic terminologies and definitions related to image database[3].\\\\
\textbf{Problem statement:} Given a dataset of hand-written English characters, the objective of this project is to implement hand-written character recognition using deep learning models. The goal is to explore various deep learning models and optimize their performance by tuning hyperparameters.\\\\

Research project involves implementing hand-written character recognition using a dataset of hand-written English characters. The project includes exploring different deep learning models and optimizing their performance by tuning various hyperparameters and plotting results of accuracy and loss in graphs.


To begin with,preprocessed the images using OpenCV, which likely involved converting the images to a format suitable for deep learning, such as resizing, normalization, and converting to grayscale.

Trained several deep learning models on the preprocessed dataset, starting with a convolutional neural network (CNN) which is commonly used for image classification tasks. This likely involved setting up the architecture of the CNN, including the number and type of layers, the activation functions, and other hyperparameters such as learning rate and batch size.

After training the CNN, proceeded to explore other deep learning models, such as an LSTM (long short-term memory) model combined with a CNN, followed by an RNN (recurrent neural network) and a GRU (gated recurrent unit) combined with a CNN. These models are commonly used for sequence prediction tasks and can be effective for handwriting recognition tasks.

Tuned various hyperparameters to optimize the model's performance, such as the number of layers, the number of neurons in each layer, the learning rate, the batch size, and others.

Finally,plotted the results of the accuracy and loss for each model, likely using graphs to visualize the performance of each model over time. This allowed you to compare the performance of each model and determine which one was most effective for your task.

Overall, project involved a thorough exploration of various deep learning models for hand-written character recognition, and a detailed analysis of their performance using hyperparameter tuning and visualization techniques.
 

